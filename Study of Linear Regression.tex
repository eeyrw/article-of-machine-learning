\documentclass[12pt,a4paper,draft]{ctexart}
\usepackage[utf8]{inputenc}
\usepackage{amsmath}
\usepackage{amsfonts}
\usepackage{amssymb}
\usepackage{graphicx}
\usepackage[left=2.70cm, right=2.70cm, top=2.70cm, bottom=2.70cm]{geometry}
\title{线性回归的学习研究}
\author{Yuan}
\date{\today}
\begin{document}

\maketitle
\begin{abstract}
摘要摘要
\end{abstract}	
https://www.cnblogs.com/jingwhale/p/4250296.html
\section{原理}
线性回归是指给定数据集$ D=\{(\mathbf{x}_{1},y_{1}),(\mathbf{x}_{2},y_{2}),...,(\mathbf{x}_{m},y_{m})\} $,其中$\mathbf{x}_i=(x_{i1};x_{i2};\\...;x_{id})$,$y_{i}\in\mathbb{R} $。线性回归试图学得$f(\mathbf{x}_{i})=\mathbf{w}^T\mathbf{x}_{i}+b$,使得$f(\mathbf{x}_{i})$接近$ y_{i}$,$\mathbf{w}$及$b$为参数,$\mathbf{w}^T=(w_1,w_2,...,w_d)$。
\subsection{最小二乘法(Ordinary Least Squares)}
为了便于计算,讲w和b合并改写为向量:
\[ \hat{\mathbf{w}}=\begin{pmatrix}
\mathbf{w} \\ 
b
\end{pmatrix}  \]
同时把数据集D表示为一个m*(d+1)大小的矩阵X,其中每一行对应一个实例,该行d个元素对应实例的d个属性值,最后一个元素恒为1:
\[ 
\mathbf{X}=
\begin{pmatrix}
	x_{11} & x_{12} & \cdots & x_{1d} & 1 \\ 
	x_{21} & x_{22} & \cdots & x_{2d} & 1 \\ 
	\vdots & \vdots & \ddots & \vdots & \vdots \\ 
	x_{m1} & x_{m2} & \cdots & x_{md} & 1
\end{pmatrix} =\begin{pmatrix}
\mathbf{x_1}^T & 1 \\ 
\mathbf{x_2}^T & 1 \\ 
\vdots & \vdots \\ 
\mathbf{x_m}^T & 1
\end{pmatrix} 
\]
同时令:
\[ \mathbf{y}^T=(y_1,y_2,\cdots,y_m) \]
损失函数是: 
\[ J(\mathbf{\hat{w}})=\min_{\mathbf{\hat{w}}}\left \| \mathbf{X}\mathbf{\hat{w}}-y \right \|^{2} \]
最小乘法的证明
对于上面的损失函数而言,J(w)显然是非负的,其去最小值的充要条件就是$ \left \| Xw-y \right \|^{2} $的倒数为0时的w的取值,从矩阵的角度来写这个式子: 
\[ \left \| Xw-y \right \|^{2}=(Xw-y)^T(Xw-y) \]
得到其关于$ \mathbf{w} $的导数为: 
\[ X^T(Xw-y) + X^T(Xw-y)=0 \]
即: 
\[ w=(X^TX)^{-1}X^Ty \]
这里,我们将\[ X^+\equiv(X^TX)^{-1}X^T \]定义为X的伪逆矩阵。

这个是一个非常好的结果,它说明使用最小二乘法,从数学上是可以直接通过公式来求解出参数w的。

注意:最小二乘法对$ \omega $的估计,是基于模型中变量之间相互独立的基本假设的,即输入向量x中的任意两项$ x_{i} $和$ x_{j} $之间是相互独立的。如果输入矩阵X中存在线性相关或者近似线性相关的列,那么输入矩阵X就会变成或者近似变成奇异矩阵(singular matrix)。这是一种病态矩阵,矩阵中任何一个元素发生一点变动,整个矩阵的行列式的值和逆矩阵都会发生巨大变化。这将导致最小二乘法对观测数据的随机误差极为敏感,进而使得最后的线性模型产生非常大的方差,这个在数学上称为多重共线性(multicollinearity)。在实际数据中,变量之间的多重共线性是一个非常普遍的现象,其产生机理及相关解决方案在“特征选择和评估”中有介绍。
\subsection{在线学习算法}
像OLS算法这种,一次性就将训练数据集中的N个数据全部都用于$ p(t|X,w,\sigma^2) $的估计中,然后通过求导得到参数的估计结果。这种算法的模型,称为批处理技术(Batch techniques),其主要缺点就在于,面对大的数据集的时候,其计算成本会非常的大。

当数据集特别大的时候,这种批处理技术就不再适用,就有一种称为序列学习算法(Sequential learning),也叫作在线算法(on-line algorithms)的算法得到了广泛的应用。

说得简单一点,序列学习算法就是将数据的样本点,看成在时间上有序的,即,数据集合不是一次性得到全部的数据,而是一个样本一个样本采集得到的。那么,在训练算法的时候,也就可以一个样本一个样本的输入的训练算法的中,利用新的样本不断的改进模型,进而达到学习的目的。

梯度下降法
有前面可以知道,线性回归的损失函数为:

\[ J(w)=\min_{w}\left \| Xw-y \right \|^{2} \]
其关于参数w梯度为:

\[ \nabla J(w)=2X^T(Xw-y) \]
梯度下降法就是:

\[ w_{k} = w_{k-1}- \rho_{k} \nabla J(w) \]
通过迭代,求得$ w_{k} $
随机梯度下降法
随机梯度下降法,就是这里要说的在线学习算法,它不像上面的梯度下降法,每次迭代都使用了全部的数据集,这里每次迭代仅仅使用一个样本:

\[ w_{k} = w_{k-1}- \rho_{k} x_{k}(x_{k}^Tw_{k-1} - y_{k}) \]
这里需要说明的是,$ \rho_{k} $ 必须满足两个条件:

\[ \sum_{k=1}^{\infty}\rho_{k}\to\infty \\
\sum_{k=1}^{\infty}\rho_{k}^2 < \infty \]
这个过程,其实是根据新的输入样本,$ 对w_{k} $的一个不断的修正的过程,一般来说,我们会把$ \rho_{k}$后面的式子称为修正量。

折两个条件,保证了在迭代过程中估计的修正值,会趋近于零。因此,对于足够大的k而言,迭代会突然停止,但是,由于有第一个条件的存在,这个停止不会发生的太早,并且不会再离结果非常远的时候,就停止了迭代。上面的第二个条件保证了,对于变量的随机性噪声,噪声的累积保持有限。

\section{优劣分析}
\section{主要用途}


\end{document}