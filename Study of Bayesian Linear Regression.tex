\documentclass[12pt,a4paper,draft]{ctexart}
\usepackage[utf8]{inputenc}
\usepackage{amsmath}
\usepackage{amsfonts}
\usepackage{amssymb}
\usepackage{graphicx}
\usepackage[left=2.70cm, right=2.70cm, top=2.70cm, bottom=2.70cm]{geometry}
\title{贝叶斯线性回归的学习研究}
\author{Yuan}
\date{\today}
\begin{document}

\maketitle
\begin{abstract}
摘要摘要
\end{abstract}	
https://www.cnblogs.com/jingwhale/p/4250296.html
\[ p(w|D)=\frac{p(D|w)p(w)}{\int p(D|w)p(w)dw} \]
\section{原理}
基于上面的讨论,这里就可以引出本文的核心内容:贝叶斯线性回归。贝叶斯线性回归不仅可以解决极大似然估计中存在的过拟合的问题,而且,它对数据样本的利用率是100\%,仅仅使用训练样本就可以有效而准确的确定模型的复杂度。

这里面对的模型是线性回归模型,其详细的介绍可以参见前面的文章《线性回归》,线性回归模型是一组输入变量x的基函数的线性组合,在数学上其形式如下: 
\[ y(x,w)=w_{0}+\sum_{j=1}^{M}\omega_{j}\phi_{j}(x) \]
这里$ \phi_{j}(x) $就是前面提到的基函数,总共的基函数的数目为M个,如果定义$ \phi_{0}(x)=1 $的话,那个上面的式子就可以简单的表示为: 
\[ y(x,w)=\sum_{j=0}^{M}\omega_{j}\phi_{j}(x)=w^T\phi(x)\]
\[ w=(w_{0},w_{1},w_{2},...,w_{M})\]
\[ \phi=(\phi_{0},\phi_{1},\phi_{2},...,\phi_{M}) \]

则线性模型的概率表示如下:
\[ p(t|x,w,\beta)=N(t|y(x,w),\beta^{-1}I) \]
假设参数w满足高斯分布,这是一个先验分布:

\[ p(w)=N(w|0,\alpha^{-1}I) \]
一般来说,我们称p(w)为共轭先验(conjugate prior)。这里t是x对应的目标输出,$ \beta^{-1} $和$ \alpha^{-1} $分别对应于样本集合和w的高斯分布的方差,w是参数,

那么,线性模型的对数后验概率函数:

\[ ln p(\theta|D)=ln p(w|T)=-\frac{\beta}{2}\sum_{n=1}^N\{y(x_{n},w)-t_{n}\}^2+\frac{\alpha}{2}w^Tw+const \]
这里T是数据样本的目标值向量,$ T=\{t_1,t_2,...,t_n\} $,const是和参数w无关的量。关于这个式子的具体推导,可参见前面的文章《多项式回归》。
\section{优劣分析}
\noindent \textbf{优点:} 
\begin{itemize}
	\item 贝叶斯回归对数据有自适应能力,可以重复的利用实验数据,并防止过拟合 
	\item 贝叶斯回归可以在估计过程中引入正则项 
\end{itemize}
\noindent \textbf{缺点:} 
\begin{itemize}
	\item 贝叶斯回归的学习过程开销太大
\end{itemize}


\section{主要用途}


\end{document}